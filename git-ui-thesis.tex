% git-ui-thesis.tex
% Created: 3/4 2015

\documentclass[12pt,a4paper,article,compsoc]{IEEEtran}
\hyphenation{op-tical net-works semi-conduc-tor}
\usepackage{paralist}
\usepackage{cite}

\begin{document}
	% Settings of author and title
	\title{This title is not determined yet}
	\author{Robin~Olofsson and Sebastian~Hultstrand}
	%\IEEEcompsocitemizethanks{\IEEEcompsocthanksitem R. Olofsson and S. Hultstrand are students are students at the Software engineering department at Blekinge Tekniska Högskola, Blekinge, Sweden.}
	
	% Abstract and keyword section
	\IEEEcompsoctitleabstractindextext{%
		\begin{abstract}
			This should be and abstract.
		\end{abstract}
		\begin{IEEEkeywords}
			Keywords separated by commas.
		\end{IEEEkeywords}}
		
		% Title
		\maketitle
		
		% Content
		\section{Introduction}
		TODO Skriva grejer
		
		% Background
		\section{Background}
		
		% Research methodology
		\section{Research methodology}
			\subsection{Research questions}
			\begin{enumerate}
				\item What is most commonly used when getting started with git, CLI or GUI?
				\item What is most commonly used by experienced git users, CLI or GUI? (May be the same answer as Q1)
				\item What can be gained from using the GUI, in the aspects of functionality, convenience and usability?
				\item What can be gained from using the CLI, in the aspects of functionality, convenience and usability?
				\item At what stage of git proficiency would you stand to gain the most from the CLI or the GUI? (Looking at what should you start out with to learn how git works and how it is used, and what you will gain the most from when you get to a higher level of understanding git.)
			\end{enumerate}
		
			\subsection{Approach for answering research questions}
			To answer these questions we conducted a literature study to find out how git works, Git work-flows and discussions about the command line interface and the graphical user interface. Based on what is found, a design of a survey and a number of  questions for interviews can be created.\\
			The intention of the survey is to collect information about current users preferences and perceptions. We want to know what they are currently using (either CLI or GUI tool), how long they have been using it, what their perception of the two different types of tools is, how they work with git, how they began using it and how they learned to use it. This to find correlations between their experience with and use of git, to their preferences and reasoning behind their choice of tool.\\
			The interviews will be based on more open questions where the aim is to allow the subject to elaborate around why they have chosen to work like they do, and discuss the perks of the two different types of interfaces that we put forward. The aim of the exercise is to get a deeper understanding of the reasoning that goes into the choice, an understanding that the survey will not be able to supply.\\
			Thirdly we will also be looking at examples of the UIs to make an assessment of the level of usability and user friendliness. We will look at the CLI tool in a Linux environment, the GUI based tools Sourcetree and Smart GitHg, both of which are available for the three large platforms (Windows, Mac and Linux). % TODO Add source for this.
			\\\\
			In summary questions 1 and 2 will be answered in the survey where questions pertaining to what UI the subject is currently using and what UI the subject started using Git in. Correlations will also searched for in the 2012 GitSurvey \cite{GitUserSurvey} to see how the results compare to other studies. Questions 3 and 4 will be answered in part through some of the perception related questions in the survey coupled with the interviews and most of all the usability assessment conducted on the UIs mentioned above. Finally question 5 will be answered by, in part, statistics from the survey. Looking at questions pertaining to perception of the UIs, time consumption associated with the subjects use of git, both general and problem based, and the subjects varying use of and proficiency with git. Partly through the interviews where being able to hear what drew the subjects to the UI that they use, what they believe to be important in the choice and their thoughts about what was found during the usability assessment.
		 
			\subsection{Literature review design}
			
			\subsection{Empirical study design}
			
				\subsubsection{Survey}
				\subsubsection*{Goal with the survey}
				The survey consists of 13 multiple choice questions, the questions are aimed to be as easy and quick to answer as possible in order to persuade people to answer the survey. What the survey is aimed at finding out is the following; the level of experience the subject has with git, what UI they are currently using, what they have used in the past and if there is a specific reason for why they use the interface that they use. We also want to find out how the subject learned to use git, what they primarily use Git for, what operating system they are working on and how much time they spend on average interacting with Git as part of their development work. Finally we want to get some insight as to how often they need to consult another person or resources online or in handbooks to solve issues that arise in Git, accompanied with some word association with the two different UI types to get a statistical view of how the users perceive them.
					
				\subsubsection*{Questions and alternatives}
				\begin{enumerate}
					\item How long have you been working with git?\\
					\begin{inparaenum}[\itshape a\upshape)]
						\item \textless 6 months
						\item 6 months - 1 year
						\item 1-3 years
						\item 3-5 years
						\item 5-10 years
					\end{inparaenum}
					\item How would you rate your knowledge of git?\\
					\begin{inparaenum}[\itshape a\upshape)]
						\item Novice
						\item Amateur
						\item Average
						\item Pro
						\item Master
					\end{inparaenum}
					\item What type of UI are you primarily using for git?\\
					\begin{inparaenum}[\itshape a\upshape)]
						\item Mainly CLI
						\item Mainly GUI
						\item Both equally
					\end{inparaenum}
					\item How did you learn to use git?\\
					\begin{inparaenum}[\itshape a\upshape)]
						\item Colleague or friend
						\item Open source community
						\item Online tutorials
						\item Book
						\item Course
						\item Self taught
					\end{inparaenum}
					\item What interface did you use when starting out with git?\\
					\begin{inparaenum}[\itshape a\upshape)]
						\item CLI
						\item GUI
					\end{inparaenum}
					\item Why do you use the interface you use today?\\
					\begin{inparaenum}[\itshape a\upshape)]
						\item Personal preference
						\item Standards for work environment
						\item Outside influences
						\item It is what i know best
					\end{inparaenum}
					\item What operating system do you primarily use?\\
					\begin{inparaenum}[\itshape a\upshape)]
						\item Windows
						\item Mac
						\item Linux
					\end{inparaenum}
					\item What do you mainly use git for?\\
					\begin{inparaenum}[\itshape a\upshape)]
						\item Basic change tracking
						\item Collaborative work
						\item Large scale version control with branch management and code reviewing through git
					\end{inparaenum}
					\item How minutes per hour of development work on average is spent on interfacing with git?\\
					\begin{inparaenum}[\itshape a\upshape)]
						\item \textless 5min
						\item 5-10 min
						\item 10-15 min
						\item 15-20 min
						\item 20-25 min
						\item 25-30 min
						\item \textgreater 30 min
					\end{inparaenum}
					\item How many times during a work day do you need to seek help for an issue with git?\\
					\begin{inparaenum}[\itshape a\upshape)]
						\item 0
						\item 1-2
						\item 2-4
						\item 4-6
						\item 6-10
						\item 10-15
						\item 15-20
						\item \textgreater 20
					\end{inparaenum}
					\item What words do you associated with Git GUI?\\
					\begin{inparaenum}[\itshape a\upshape)]
						\item Aesthetic
						\item Easy
						\item Bulgy
						\item Auto-magic
						\item Difficult
						\item Helpful
						\item Simple
						\item Intuitive
						\item Insightful
						\item Hard
						\item Time consuming
						\item Cool
						\item Professional looking
						\item Control
						\item Difficult
						\item Ugly
						\item Old fashioned
					\end{inparaenum}
					\item What words do you associated with Git CLI?\\
					\begin{inparaenum}[\itshape a\upshape)]
						\item Aesthetic
						\item Easy
						\item Bulgy
						\item Auto-magic
						\item Difficult
						\item Helpful
						\item Simple
						\item Intuitive
						\item Insightful
						\item Hard
						\item Time consuming
						\item Cool
						\item Professional looking
						\item Control
						\item Difficult
						\item Ugly
						\item Old fashioned
					\end{inparaenum}
				\end{enumerate}
				
				\subsubsection*{Crafting and distribution}
				The survey will be made as a web form that we can easily collect the data from and most of all distribute easily. Distribution of the survey is done through a number of different channels online such as BTH:s student mail list, through contacts at different Swedish companies, trough large companies that work with git that we have been able to procure assistance with and finally 3-7 [EXACT NUMBER LATER] different sub-reddits (Forum boards at the popular web forum Reddit) associated with software development.
				
				\subsubsection*{Analysis of survey}
				The survey is crafted so that it can be evaluated and analyzed in the following way. 
				
				\subsubsection{Interviews}
				\subsubsection*{Interview format}
				The format of the interviews will be a discussion on the topic of this paper where the aim is to get the subjects to explain their standing on the issue of CLI v. GUI in git interaction, and most of all elaborate around perceptions and experiences that they have. We will also supply some information about the UI:s to allow the sessions to go into as much detail as possible. 
				We will conduct 4-7 [NUMBER TO BE DETERMINED LATER] interviews with a number of students new to git and a number of professionals using it every day, the sessions will be recorded and transcribed later to minimize the loss of information.
					
				\subsubsection*{Questions and topics to discuss}
		
		% Literature study		
		\section{Literature review}
		
			\subsection{Command-line versus Graphical user interface}
			Based on what we found people with experience about Git will in most, if not all cases prefer the command-line interface over the graphical user interface.\\
			The reason behind this is however, not crystal clear.In the 2012 edition of GitSurvey \cite{GitUserSurvey}, about 97\% of the people who answered the question “What Git implementations do you use?”, use the git command-line whilst only 8\% use some graphical user interface.\\\\
			If we go back to a time, before git. A time when the concept of graphical user interfaces were quite new, there was a study conducted on students to see whether they would prefer the command-line interface over the graphical user interface. The study was done at Waikato University, New Zealand, back in 1992 and involved a mix of Macintoshes and IBM PC compatibles.\\
			A group of students were placed in front of these machines so that the staff was able to observe how they responded to interacting towards Macintoshes graphical interface and IBM PC compatibles command-line.\\
			The theory was that the students, whom had no prior computer experience, would prefer the Macintosh because of the fact that a graphical user interface has a less harsh learning curve than a command-line.This theory was also backed up by Morgan, et al \cite{MouseToRat} and Kirkpatrick et al \cite{MacVsWindows}.\\
			Based on this study and another, more git-focused study that was conducted in 2014 \cite{GitInClassroom}, the theory seems to be that graphical user interfaces would appeal to novice users more.\\
			However theoretically, as users get more and more experienced they will gradually start to prefer the command-line interface over a graphical user interface because they are able to complete their tasks faster and to maintain more control over what they are doing and what's going on.\\\\
			Now, lets forget about the theories and focus on what these studies yielded in the form of results. Generally speaking it would seem that a user tends to use what they find to be easier \cite{Treweek}. Because graphical user interfaces do have a more lenient learning curve it would therefore appear logical that a user would prefer that over a command-line interface. In terms of Git, graphical user interfaces are likely more valuable when the knowledge of the system matures for the user \cite{GitInClassroom}. Git is complex, if you check the answers for the question “What do you hate about Git?” in the 2012 edition of GitSurvey you will notice that this is one of the answers. Another answer to this question is ”requires steep learning curve for newbies”. This is what spurred the development of graphical user interfaces for Git, the goal was to hide some of Git:s complexity. \cite{WrongWithGit}
			
			\subsection{Summary of the literature study results}
			The graphical user interfaces that are available for Git is a lot less used than the Git command-line. Yet people with low experience seem to prefer the graphical user interfaces because they appear to be easier to use. The people that answered the GitSurvey in 2012 was a group of mostly everyday users or more experienced than average users of Git \cite{GitUserSurvey}, so it is hard to judge whether the statement in the line above is valid or not but if you are to believe the previously introduced studies then it would make sense. After looking at the previous studies, we would have to say that command-line interfaces seem to be something that attracts more experienced users whilst graphical user interfaces attracts novice users.
		
		% Discussion	
		\section{Discussion}
		
		% Conclusion
		\section{Conclusion}
			
		
		% Acknowledgment section
		\section*{Acknowledgment}
		
		% References section
		\newpage
		\begin{thebibliography}{1}
			
			\bibitem{GitUserSurvey}
			GitSurvey2012 - Git SCM Wiki. (2012). GitSurvey2012 - Git SCM Wiki. Available: http://git.wiki.kernel.org/index.php/GitSurvey2012. Last accessed 4th April 2015..
			
			\bibitem{MouseToRat}
			Morgan, K., Morris, R. \& Gibbs, S. (1991). When does a Mouse become a Rat? or … Comparing Performance and Preferences in Direct Manipulation and Command Line Environment. The Computer Journal. 34 (3), p265-271.
			
			\bibitem{MacVsWindows}
			Kirkpatrick, D., Sherman, S., \& Deutschman, A. (1993). Mac vs Windows. Fortune Internc\#ional. 128 (8), p57-64.
			
			\bibitem{GitInClassroom}
			Kelleher, J. (2014, 17-19 Jan). Employing git in the classroom. Paper presented at 2014 World Congress on Computer Applications and Information Systems (WCCAIS). Retrieved 4th April, 2015, at IEEE Xplore. DOI: 10.1109/WCCAIS.2014.6916568.
			
			\bibitem{Treweek}
			Treweek, P. (1996). Comparing Interfaces: Should We Assume that Ease of Use Influences Users' Preference?. Presented at Sixth Australian Conference on Computer-Human Interaction. Retrieved 4th April, 2015, at IEEE Xplore. DOI: 10.1109/OZCHI.1996.560004.
			
			% This is not being used as far as i know.
			%\bibitem{Hart}
			%Hart, D. (2009). A survey of source code management tools for programming courses. Journal of Computing Sciences in Colleges. 24 (6), p113-114.
			
			\bibitem{WrongWithGit}
			Perez De Rosso, S. \& Jackson, D. (2013). What's wrong with git?: a conceptual design analysis. Presented at 2013 ACM international symposium on New ideas, new paradigms, and reflections on programming \& software. Retrieved 4th April, 2015, at ACM Digital Library. DOI: 10.1145/2509578.2509584.
			
		\end{thebibliography}
\end{document}